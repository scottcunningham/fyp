%% frontmatter.tex
%%

\title{Dissipation in Deforming Chaotic Billiards}
\author{Alexander Harvey Barnett}
\degreemonth{October} % month final submission occurs.
\degreeyear{2000}
\degree{Doctor of Philosophy}
\field{Physics}
\department{Physics}
\advisor{Eric J. Heller} % Category I added.

\maketitle
\copyrightpage


\begin{abstract}
% limited to 1.5 pages, double-spaced (Registrar's Office guidelines).
% Also limited to 350 words. I claim $\mu \sim \omega^4$ is a single word.

Chaotic billiards (hard-walled cavities)
in two or more dimensions are paradigm systems in the fields
of classical and quantum chaos.
We study the dissipation (irreversible heating) rate in such billiard systems
due to general shape deformations which are periodic in time.
We are motivated by older studies of one-body nuclear dissipation
and by anticipated mesoscopic applications.
We review the classical and quantum linear response theories of dissipation rate
and demonstrate their correspondence in the semiclassical
limit.
In both pictures, heating is a result of stochastic energy spreading.
The heating rate can be expressed as a frequency-dependent
friction coefficient $\mu(\omega)$, which
depends on billiard shape and deformation choice.
We show that there is a special class of deformations for which
$\mu$ vanishes as like a power law in the small-$\omega$ limit.
Namely, for deformations which cause translations and dilations
$\mu \sim \omega^4$ whereas for those which cause rotations $\mu \sim \omega^2$.
This contrasts the generic case for which $\mu \sim \omega^0$.
We show how a systematic treatment of this special class leads to
an improved version of the `wall formula' estimate for $\mu(0)$.

We show that the special nature of dilation (a new result)
is semiclassically equivalent to a
quasi-orthogonality relation between the (undeformed) billiard quantum
eigenstates on the boundary.
This quasi-orthogonality forms the heart of a `scaling method' for the numerical
calculation of quantum eigenstates, invented recently by Vergini and Saraceno.
The scaling method is orders of magnitude more efficient than any other
known billiard quantization method, however an adequate explanation for its
success has been lacking until now.
We explain the scaling method, its errors, and applications.
We also present improvements to Heller's plane wave method.

Two smaller projects conclude the thesis.
Firstly, we give a new formalism for quantum point contact (QPC)
conductance in terms of scattering
cross-section in the half-plane, of use in open mesoscopic and atomic systems.
We derive the maximum conductance through a tunneling QPC coupled to a resonator.
Secondly, we numerically model a novel design of
coherent atom waveguide which uses the dipole
force due to evanescent light fields leaking from
an optical waveguide mounted on a substrate.

\end{abstract}

%We show how a systematic subtraction of the `special' components of a general
%deformation can be used to give an improved
%version of the `wall formula' estimate for $\mu(0)$.
%We believe this is the first study of $\omega$-dependent heating rate in
%billards, and the first consideration of the `special' nature of dilation.



\newpage
\addcontentsline{toc}{section}{Table of Contents}
\tableofcontents

% these are optional in the Jan 2000 Harvard thesis GSAS guide:
%\listoffigures
%\listoftables
%(Cut them for my personal thesis format).

% cccccccccccccccccccccccccccccccccccccccccccccccccccccccccccccccccccccccccc
\begin{citations}

\vspace{0.8in}

\ssp
\noindent
Large portions of Chapters~\ref{ch:dil} and \ref{ch:wall}, as well as some
of Sections~\ref{sec:qcc} and \ref{sec:quasi}
have appeared in the following two papers:
\begin{quote}
	``Deformations and dilations of chaotic billiards:
	dissipation rate, and quasi-orthogonality of the boundary
	wavefunctions'',
	A. H. Barnett, D. Cohen, and E. J. Heller,
	Phys. Rev. Lett. {\bf 85}, 1412 (2000), {\tt nlin.CD/0003018};
	\vspace{.1in} \\
	``Rate of energy absorption for a driven chaotic cavity'',
	A. H. Barnett, D. Cohen, and E. J. Heller,
	submitted to J. Phys. A, {\tt nlin.CD/0006041}.
\end{quote}
The numerical methods of
Chapter~\ref{ch:verg} were used to calculate data appearing in the above
papers and in the following:
\begin{quote}
	``Parametric evolution for a deformed cavity'',
	D. Cohen, A. H. Barnett, W. Bies, and E. J. Heller,
	submitted to Phys. Rev. E, {\tt nlin.CD/0008040}.
\end{quote}
Chapter~\ref{ch:qpc} appears in its entirety as
\begin{quote}
	``Mesoscopic scattering in the half-plane:
	how much conductance can you squeeze through a small hole?'',
	A. H. Barnett, M. Blaauboer, A. Mody, and E. J. Heller,
	submitted to Phys. Rev. B,
	{\tt cond-mat/0008279}.
\end{quote}
Finally, most of Chapter~\ref{ch:atom} has been published as
\begin{quote}
	``Substrate-based atom waveguide using guided two-color
	evanescent light fields'',
	A. H. Barnett, S. P. Smith, M. Olshanii, K. S. Johnson,
	A. W. Adams, M. Prentiss,
	Phys. Rev. A {\bf 61}, 023608 (2000), {\tt physics/9907014}.
\end{quote}
Electronic preprints (shown in {\tt typewriter font}) are available
on the Internet at the following URL:
\begin{quote}
	{\tt http://arXiv.org}
\end{quote}
\end{citations}




\begin{acknowledgments}

Completing this doctoral work
has been a wonderful and often overwhelming experience.
It is hard to know whether it has been grappling with the physics
itself which has been the real learning experience, or grappling with
how to write a paper, give a coherent talk, work in a group, teach section, code
intelligibly, recover a crashed hard drive,
stay up until the birds start singing,
and... stay, um... focussed.

I have been very privileged to have undoubtedly the most intuitive,
smart and supportive advisor anyone could ask for, namely Rick Heller.
Ever since I learned from him what an avoided crossing was
(animated in full colour, naturally), I have been stimulated and excited
by his constant flow of good ideas.
Rick has an ability to cut through reams of algebra with a single visual
explanation that I will always admire, and I have learned a great deal
of physics from him.
He has fostered certainly the most open, friendly,
collaborative and least competitive
research group in the theory wing of this Department.
He has also known when (and how)
to give me a little push in the forward direction when I needed it.

Throughout my six years, I was supported for many semesters
by the National Science Foundation, and the Institute for Theoretical
Atomic and Molecular Physics (ITAMP), through the generosity of my advisor.
During my first year I thank the Kennedy Memorial Trust in London for
providing my funding.

Rick's other students and post-docs, both past and present, comprise a
superb research group.
The ability to bounce ideas off so many excellent minds has been priceless.
My most intense collaboration has been with Doron Cohen,
whose clarity, persistence, ability to create new models, and
ability to write a new publication every month,
has taught me a lot.
(I am still working on the publication per month aspect.)
Michael Haggerty has similiarly influenced me, and I
think I can safely say that everyone in the group has benefitted from
his generosity and collaboration.
In my first couple of years in the Heller group, Adam Lupu-Sax and I
swapped many thoughts on numerical methods, while Lev Kaplan prevented me
from getting too scarred by the new realm of quantum chaos.
Jesse Hersch shared many good times and wacky science experiments
with me, and introduced me to a world of
electrostatically-levitated chalk particles.
Areez Mody has been a companion and goldmine of mathematical knowledge
since the year we arrived together.
Scot Shaw has been my numerical methods partner-in-crime.
Bill Bies was my buddy through thesis formatting hell.
It has also been my pleasure to work with (and hang out with)
Ji\v{r}\'{\i} Van\'{\i}\v{c}ek,  Axel Andr\'{e},
Miriam Blaauboer,
Ragnar Fleischmann,  Greg Fiete,  Allison Kalben, Natasha Lepore,
Michael Efroimsky (and his stories),
Manny Tannenbaum (and his accents), Sheng Li and Stephan Filipov;
also from the earlier days
Maurizio Carioli, Allan Tameshtit,
Jonathan Edwards, Kazuo Hirai (and his valves),
Bill Hoston,  Neepa Maitra,  Martin Naraschewski
and Steve Tomsovic (and his jazz).

Our group's secretary Carol Davis is surely the kindest, coolest
and most witty person one could possibly hope to spend a lunch-break
gossiping with. She has also helped me out many a time during those
pre-presentation panics.
I must also thank Mary Lampros for her unique blend of caring and total
organisation. She will be sorely missed in this Department.

On the foreign front,
I must thank Eduardo Vergini for fascinating email exchanges, and
guiding me towards an understanding of
the incredible numerical method he invented along with Marcos Saraceno.
From the group of Mara Prentiss, Steve Smith and Maxim Ol'shanii
were always there to fill me in on the practice and theory of atom optics.

There are countless others who have
been there for me throughout my time as a graduate student.
From the world of physics,
Meredith Betterton, Joseph Thywissen and Rosalba Perna
stand out as both good, caring friends and providers of
fascinating physics conversations.
From other worlds, Erika Evasdottir and my
housemates Vitaly Napadow and John Iversen have been wonderful
influences and friends; they have lead me from deconstruction to drumming.

My fascination with the physical world is undoubtedly due to
the influence of my father, Ross. He showed me the joys of
high-dispersion prisms in sunlight at an early age,
and let me spend most of my childhood
building Tesla coils and playing with BASIC on our 6502 home computer.
Even now, he proofreads my papers.
My late mother Pat taught me a love of language, of art,
of teaching, and of signs,
and it is sad to realise that she cannot be present for this moment in my life.
The same realisation holds true for my grandfather Oliver, who
is highly responsible for my tinkering nature.
My courageous sister Jess has kept me sane
during my hardest times via the transatlantic phone system.

Finally, Liz Canner has been my guiding light and love
over these last two years. She has seen my best and my worst,
and provided support, hugs, and taken me places I never imagined.
Even when my emotional and research brains became so hopelessly entwined
that I dreamt that the two of us were overlapping eigenfunctions,
she still loved me. And she even thought it was cute.

\end{acknowledgments}





%ddddddddddddddddddddddddddddddddddddddddddddddddddddddddddddddddddddddddddd
\dedication

\begin{quote}
\hsp
\em
\raggedleft

Dedicated to my father Ross,\\
my late mother Pat,\\
and my sister Jess.

\end{quote}


\newpage

\startarabicpagination

%%% end

