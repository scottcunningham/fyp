\documentclass{article}

\title{An Anonymising Web Serving Network with Community-Driven Censorship}
\begin{document}

\author{Scott Cunningham}
\date{Supervisor: Stephen Barrett}

\maketitle{}

\bibliographystyle{plain}%Choose a bibliograhpic style

\newpage{}

%---------------------------------------------------------
\tableofcontents{}

\newpage{}

%---------------------------------------------------------
%---------------------------------------------------------
\section*{Abstract}

\newpage{}
%---------------------------------------------------------
\section{Introduction}
%

The world-wide web as is exist today is inherently centralised. The common case for web-serving is
that when a user wants to view a certain website X, they contact a server known to be hosting X and
download this web page directly from this central server. This model changes slightly when websites
use Content Delivery Networks in order to distribute copies of their website across several
geographic locations in order to improve page-load times for users in that location. Regardless,
this model is also centralised.

This centralisation is the cause of several concerns, these being:
\begin{itemize}
    \item{Privacy concerns. For example, it is possible for a third-
          party listener on a network to inspect the packets coming from a user and tell which IP address
          this user is connecting to. This is problematic because it allows third parties to infer which web
          sites a user is visiting, and invade users' privacy by profiling their internet activity.}
    \item{This centralisation also leaves these central website servers open to attack. People who find the
content of a site objectionable can lead large distributed attacks against web servers, referred
to as a "DDoS" (Distributed Denial of Service).}
    \item{bad censorship, not decided by public, decided by those who are in power}
mention anonymous publishing

\end{itemize}

This project aims to address the issues outlined above by developing a system in which web
serving is decentralised anouisadoijd etc

\section{Design}

%---------------------------------------------------------
\section{Implementation}

%---------------------------------------------------------
\section{Evaluation}

%---------------------------------------------------------
\section{Further work}

%---------------------------------------------------------

testing references: blah blah blah ~\cite{blanchfield}. nblah blaqh ~\cite{levacher} ~\cite{freenet}.

asdf

~\cite{torrent}

%---------------------------------------------------------
\section{Methods/Algorithms}

Goals:

Durability - no single point of failure
Reader anonymity
Publisher anonymity
Censorship


Ways to do things:

Durability
    - We distribute things around a decentralised network
    - No single point of failure
    - On top of Kademlia (optional)
        - We split pages into parts
        - Need to take down several hosts to get file offline
        - Not taking cache into account
Reader anonymity
    - Based on Levacher - randomly change source to self on hops
Publisher anonymity
    - Don't rely on often-used crypto key
    - Need to make it seem like a "publish" command doesn't come from self
    - TODO - flesh out implementation ideas here
Don't want to keep items online forever
    - like freenet
    - set long-lasting timer when item created
    - downvotes contribute to death rate of content
    - not downvoting -> keep it on for longer
    - MENTION LEVACHER - LAZINESS 
Censorship
    - Can be done in several ways
    - Evaluate and weigh pros/cons of all
    - one style: lavacher ripple effect
    - other: refuse to forward packets
    - algorithm for user holding data item
        - decomposition algorithm
        - downvote based

    - 

%---------------------------------------------------------
\section{Conclusions and future work}

%---------------------------------------------------------
%---------------------------------------------------------
\newpage{}
\bibliography{bib}
\end{document}
