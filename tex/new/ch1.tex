%% intro.tex
\chapter{Introduction and summary}
%%%%%%%%%%%%%%%%%%%%%%%%%%%%%%%%%%%%%%%%%%%%%%%%

\subsubsection{Structure of this thesis}


This thesis falls naturally into four parts, which are relatively
independent:
\begin{itemize}
\item Study of dissipation rate in deformed chaotic billiards (Chapters~2,
3 and 4),
\item Improved numerical methods for quantization of billiards
(Chapters~\ref{ch:ipwdm} and \ref{ch:verg}),
\item Half-plane scattering
%theory
approach to mesoscopic conductance
(Chapter~\ref{ch:qpc}), and
\item Design of an atom waveguide using two-color evanescent light fields
(Chapter~\ref{ch:atom}).
\end{itemize}

The first two parts form the main body of the thesis, and they are both
devoted to the study of billiard systems (hard-walled cavities
enclosing a region of free space)
in which the classical motion is chaotic.
The quantum mechanics of such systems has become known as the field of
`quantum chaos'.
The first part probably contains the most significant new physical results;
this is reflected in the choice of thesis title.
The second part can be viewed merely as a description of numerical
quantum-mechanical calculations that play a
supporting role in the first part.
However, there will also turn out to be a surprising reciprocal connection,
namely that results from the first part will provide a much-needed
explanation for the success of a very efficient numerical technique
in the second part.
The intertwining of these two subject areas had turned out to be one of the
most beautiful surprises in this body of research.

The third and fourth parts form essentially separate projects, and can
therefore be read independently.
However they do share with the rest of the thesis the common theme of
wave mechanics: the third presents a new approach to
the transport of quasiparticle waves in mesoscopic systems,
and the fourth models confined electromagnetic waves to trap and guide atoms
(which themselves can be treated as coherent matter waves).

The goals and subject matter of the four parts are sufficiently different
to merit individual introductions and summaries, which now follow without
further ado.



% 1111111111111111111111111111111111111111111111111111111111111111111111111111
\subsubsection{Chapters 2,3 and 4: Dissipation rate and
deformations of chaotic billiards}


The dynamics of a particle inside a cavity
(billiard) in $d=2$ or 3 dimensions 
is a major theme in studies of classical and quantum chaos
\cite{ottbook,hellerleshouches,berryleshouches}.
Whereas the physics of time-independent chaotic systems 
is extensively explored, less is known 
about the physics when such a system is `driven' (time-dependent
chaotic Hamiltonian).
The main exceptions are the studies of 
the kicked rotator and related systems \cite{qkr}. 
However, the rotator (with no kicks) 
is a $d=1$ integrable system, whereas we 
are interested in chaotic ($d\ge2$) cavities.


Driven cavities have been of special interest since the 1970s in
studies of the so-called `one-body' dissipation rate in vibrating nuclei 
\cite{wall,koonincl,kooninqm,jarz92,jarz93}.
A renewed interest in this problem is anticipated
in the field of mesoscopic physics. Quantum dots \cite{been,dittrich}
can be regarded as small 2D cavities whose shape 
is controlled by electrical gates. Quasiparticle motion inside the dot
can have long coherence (dephasing) times,
and enable the semiclassical regime to be approached (many wavelengths
across the system).


In Chapter~\ref{ch:review} I give tutorial review of the
theory of dissipation in general driven
ergodic systems, which is quite a young field.
The Hamiltonian is controlled by a single parameter $x$, whose
time-dependence will be $x(t)=A\sin(\omega t)$ where $A$ is the amplitude  
and $\omega$ is the driving frequency.
In both the classical (Section~\ref{sec:classreview}) and quantum-mechanical
(Section~\ref{sec:qmreview}) pictures, dissipation is a result of
{\em stochastic energy spreading}.
Once this spreading is established, the pictures can be unified \cite{doronfrc}.
Irreversible growth of energy (heating) is then a result of biased diffusion
(a random walk) in energy.
I will confine myself to a regime where linear response theory (LRT) is valid.
In the quantum case this is known as the Kubo formalism, although the
language of the energy spreading picture appears different (I connect the
two pictures in Section~\ref{sec:resp}).
The heating rate is given by
\be
\label{i_heat}
	\frac{d}{dt}\langle {\cal H} \rangle \ = \
	\mu(\omega) \cdot \half  (A\omega)^2,
\ee
where the `friciton coefficient' $\mu(\omega)$ is related
linearly\footnote{The relation depends upon the initial energy distribution.}
to $\cew$, the correlation spectrum
of the time-dependent fluctuating quantity
$-\partial {\cal H}/\partial x$.
The latter (a generalized `force' on the parameter $x$) specifies the
random `kicks' up or down in energy that particles receive.
In the quantum case $\cew$ is the `band profile' (average off-diagonal shape)
of the matrix $\partial {\cal H}/\partial x$ in the energy representation.
In Section~\ref{sec:qcc} I demonstrate theoretically and numerically the
semiclassical equivalence of the classical and quantum versions of $\cew$,
and examine quantum effects beyond the band profile.


In Chapters~\ref{ch:dil} and \ref{ch:wall} we specialize to a
system of non-interacting particles
inside a billiard whose walls are deformed by the parameter $x$.
One can specify any deformation by a function $D(\mbf{s})$,
where~$\mbf{s}$ specifies location of a wall element
on the boundary (surface) of the billiard,
and $D(\mbf{s})x$ is the resulting normal displacement of this
wall element.
We will be interested in low-frequency driving, 
meaning $\omega \ll 1/\tbl$ the mean collision frequency.
In the $\omega\rightarrow0$ limit (uniform parameter velocity $\dot{x}$),
the heating rate (\ref{i_heat}) is given by the dc friction
$\mu(0)$ proportional to
$\ve = \tilde{C}_{\tbox E}(\omega\rightarrow0)$.
An assumption of uncorrelated collisions
(the white noise approximation or WNA)
gives an estimate for $\ve$, which in turn in $d=3$
leads to the well-known
`wall formula' \cite{wall} (from the nuclear application),
\be
        \mu_{\tbox{E}} \  = \ \frac{N}{\mathsf{V}} 
        m v_{\tbox E} \oint D(\mbf{s})^2 d\mbf{s}.
\ee
This ($\omega$-independent) estimate of the friction applies to
a microcanonical ensemble of $N$ particles
with speed $v_{\tbox E}$ in a billiard volume $\mV$.

We analyze $\mu(\omega)$ numerically in 2D billiard shapes (generalized Sinai,
and Bunimovich stadium). We believe this is the first study of frequency-dependent
heating rate in billiard systems.
The chief discovery (Section~\ref{sec:spec})
is a class of deformations whose heating rate
vanishes in the $\omega\rightarrow0$
limit, like a power-law $\cew \sim \omega^\gamma$.
This holds even for billiards with strong chaos,
and goes completely against the WNA prediction.
The class of `special' deformations
turns out to be just the class which preserves the billiard shape.
For translations and dilations $\gamma = 4$ and for rotations $\gamma = 2$.
This is to be compared to the case of a generic deformation, for which
$\gamma = 0$ as the WNA prediction would predict.
We give classical explanations for the power-laws (which
rely on correlation on short timescales $\sim \tbl$).
Importantly, the special class is manifested in the quantum band profile too.
Thus the special nature of dilation, believed to be new in the literature,
corresponds to a quasi-orthogonality relation between
eigenstates on the boundary,
which in turn will be the key to the powerful
numerical method of Chapter~\ref{ch:verg}.
We also discuss (Section~\ref{sec:wnarev}) non-generic shape-dependent
effects (such as marginally-stable orbits) which may alter the power-laws
given above.

The goal of Chapter~\ref{ch:wall} is as follows:
given a general deformation $D\ofs$, in a given billiard shape,
we seek an analytical estimate of $\ve$ (and hence
$\mu(0)$). It is an exact result that $\ve$ is a quadratic form in
the function space of $D\ofs$.
The WNA
fails to take into account that $\ve$ vanishes for special deformations
(which form a linear subspace in $D\ofs$).
We show that how it is possible to systematically subtract
(project out) the `special' components of a general $D\ofs$.
Applying the WNA only to the remaining (`normal') component
gives an improved estimate of $\ve$.
We analytically and numerically justify this projection procedure, and test
the quality of the improved formula.
The quality is limited by that of the WNA estimate of the `normal' component,
which relies on the assumption of strongly chaotic motion.
However, in the generalized Sinai billiard
the improved formula is found to perform much better than the original WNA.

Our work replaces all {\em ad hoc} corrections which had been introduced \cite{wall}
to account for the intuitive result that translations and rotations cause no heating
at $\omega=0$.
We thereby clear up some inconsistent habits in the nuclear community
(Section~\ref{sec:wallhistory}).
More significantly,
the incorporation of the special nature of dilation is entirely new.

Note that the effect of interaction between the particles is 
not considered. If the mean free path 
for inter-particle collisions is large compared 
with the size of the cavity, then we expect that our 
analysis still applies. (However if the mean free path is 
much smaller, then we get into the hydrodynamic regime, where viscosity
becomes the dominant dissipative effect).

This work was performed in collaboration with Doron Cohen.
I also benefitted greatly from use of a classical billiard trajectory code
written by Michael Haggerty.



% 2222222222222222222222222222222222222222222222222222222222222222222222222222
\subsubsection{Chapters \ref{ch:ipwdm} and \ref{ch:verg}:
Improved billiard quantization methods}

The rapid development of electronic computing machines in the last 50 years
has had an impact on scientific research whose size is hard
\footnote{Here's another footnote to make sure it's separated from the first}
to
grasp\footnote{Some idea
of the rate of progress of this technology can be gained from such quotes as
``Such a machine in the hands of a competent operator can produce 400 full-length
products or 1,000 sums during an 8-hour working day''\cite{highspeed},
referring to an electromechanical desk calculator typical for scientific use in
the 1950s.
The diagonalization of an 8-by-8 matrix was a weekend-long task\cite{lanczos}.}.
The interplay between numerical simulations and theoretical models now plays a
crucial role in most areas of physics, chemistry, engineering, and other
sciences.
However, this impact would have been be drastically reduced were it not for
the parallel development of efficient numerical algorithms.
For instance, the invention of two techniques alone---the
diagonalization of dense matrices\cite{golub+vanloan},
and the Fast Fourier Transform\cite{numrec}---has allowed scientists
to handle hitherto undreamed-of problems on a daily basis.

Quantum chaos \cite{hellerleshouches} is no exception: it has relied heavily
on numerical solutions almost since its inception
(as did its forebear, classical chaotic dynamics\cite{ottbook}).
Billiards in $d=2$ (or sometimes 3) dimensions have been popular systems for
study because
use of the free-space Green's function allows formulation as a boundary problem.
Thus
quantum eigenstates can be calculated at much higher energy than with the
traditional (\eg finite element) methods which cover the entire domain.
High energies are so sought-after because most of the theoretical predictions
involve the semiclassical limit $\hbar\rightarrow0$.
In this part of the thesis I will present new and efficient
methods for finding these high-energy eigenstates.

The time-independent Schrodinger's equation in such a system is
the Helmholtz wave equation,
\be
\label{eq:i_helm}
	(\nabla^2 + k^2) \psi\ofr \ = \ 0,
\ee
with certain boundary conditions.
This problem is common to many other areas of physics and engineering
(mesoscopic devices, acoustics, elastodynamics of thin plates,
scalar electromagnetics and optics),
and there
has been some (but not that much) exchange of ideas between those communities
and that of quantum chaos.
If the boundary conditions are open then we have a scattering problem;
if closed, an eigenvalue problem.
I will be concerned only with the latter.
In Section~\ref{sec:eigreview} I present a review, and
categorize solution methods dependent on whether the basis
does not (Class A) or does (Class B) depend on energy.
Only Class B allows formulation as a boundary problem.
The pioneering chaotic eigenstate studies of
McDonald and Kaufman \cite{macdonald}
and Berry and Wilkinson \cite{berrywilk} used the Boundary Integral Method
\cite{bim1,bim2}
(BIM or BEM), while those of Heller
\cite{Heller84,hellerleshouches} used the Plane Wave Decomposition Method (PWDM,
an original technique).
Semiclassical quantization methods have also been developed based
on boundary matching \cite{berrykkr} or the surface of section
\cite{bogomolny}.
All these methods are Class B, and all require an expensive search
(`sweep' or `hunt') in energy-space for zero-determinants of a matrix.

In Chapter~\ref{ch:ipwdm} I present an original
Class B sweep method which is a simplified
version of Heller's PWDM. The problems of missing states and sensitivity to
basis size choice and matching point density have been solved,
and the efficieny increased.
The coefficient vector $\mbf{x}$ of the nearest eigenfunction to a given
wavenumber $k$ is given by the largest-eigenvalue ($\lambda_1$) solution to
\be
\label{i_geneigprob}
	\left[ G(k) - \lambda F(k) \right] \mbf{x} \ = \ \mbf{0} ,
\ee
where the matrix $G$ takes the norm in the domain,
and $F$ takes the norm of the boundary condition error (the `tension').
I show that $G$ can be expressed entirely on the boundary, and discuss
improved `hunt' methods for zeros of $\lambda_1^{-1}$ (which give the desired
eigenwavenumbers $k$). However, a few diagonalizations of (\ref{i_geneigprob})
are still required per state found.

Of much more significance is Chapter~\ref{ch:verg}.
Here I analyse the
`scaling method' of Vergini and Saraceno
\cite{v+s,verginithesis},
which despite being little-understood and little-used, is without doubt the
most significant advance in numerical billiard quantization in the last 15 years.
Eigenstates are given by the large-$\lambda$ solutions of
\be
\label{i_vergep}
	\left[ \frac{dF}{dk}(k) - \lambda F(k) \right] \mbf{x} \ = \ \mbf{0} ,
\ee
however, through a certain choice of boundary weighting function
an amazing property of $F$ and $dF/dk$ emerges:
they are {\em quasi-diagonal} (have very small off-diagonal elements)
in a basis of
the exact eigenfunctions rescaled to all have the same wavenumber $k$.
This allows (\ref{i_vergep}) to return up to $N/10$ useful eigenfunctions
for a {\em single} diagonalization,
and entirely eliminates the need for `hunt' procedures.
Here $N$ is the matrix size (semiclassical basis size).
The relative efficiency over sweep methods is $\sim 10^3$ when there are
several hundred wavelengths across the system, and moreover,
increases further with increasing $k$ and dimension $d$!
Remarkably, no adequate explanation of the key quasi-diagonality property
has been known until now.
I give, for the first time, a semiclassical
explanation in terms of the `special' nature of
the dilation deformation (from Chapter~\ref{ch:dil}).
I also correct errors in the original authors' derivation
\cite{v+s,verginithesis} of higher-order tension terms.

Both chapters are presented as a practical `how-to' guide to the diagonalization
of $d$-dimensional billiards, and I hope they may be of use
to other communities who solve the Helmholtz eigenproblem.
I thoroughly analyse the various types of error in both the sweep
and scaling methods, compare results from the two, and discuss
the use of real and evanescent plane
wave basis sets.
For illustration, I use Bunimovich's stadium billiard (a shape known to
be classically-chaotic \cite{bunimovich}), in which
evanescent basis sets have been pioneered by Vergini\cite{verginithesis}.
Currently the scaling method applies only to Dirichlet boundary conditions.
Adequate basis
sets for more general shapes is an area in dire need of future research.
My work has involved deriving a collection of useful new formulae for
boundary evaluation of domain integrals of Helmholtz solutions:
these are presented in Appendix~\ref{ap:perim}.

Two applications of the scaling method are presented in this thesis.
The first is the quantum band-profile calculations for Chapters~2--4.
The second is an efficient
evaluation of overlaps of eigenstates of a billiard with
eigenstates of the same billiard deformed by various finite amounts
(Section~\ref{sec:pnm}).
The profiles of the resulting matrices can be viewed as local densities
of states (`line shapes'), which are analysed in our publication \cite{pnm}.
The diagonalization of the deformed stadium billiard is believed to be new.

During this work,
I have benefitted much from fruitful exchanges with Eduardo Vergini.
I must thank Doron Cohen for first alerting me to the
semiclassical estimation of the band profile of matrix
elements on the boundary.
Finally,
Appendix~\ref{ap:perim} resulted from collaboration with Michael Haggerty.



% 33333333333333333333333333333333333333333333333333333333333333333333333333
\subsubsection{Chapter~\ref{ch:qpc}: Quantum point contact conductance
and scattering in the half-plane}

This third part continues the theme of wave mechanics of non-interacting
particles.
However attention shifts from closed to open systems, namely
the transport
of quasiparticles (in a 2D electron gas) though a general two-terminal
mesoscopic electronic device, or `quantum point contact' (QPC)
\cite{been,dittrich}.

We model the conductance of a QPC, in linear response.
If the QPC is highly non-adiabatic or near to
scatterers in the open reservoir regions, then
the usual distinction between `leads' and `reservoirs' breaks down.
This situation arises in the recent experimental work of
Katine \cite{kati97} and Topinka \cite{topinka}
in the Westervelt group
here at Harvard, where open resonant and scattering geometries were studied.
In such systems the Landauer formula \cite{LB1,LB2,LB3,datta,dittrich}
for the conductance
(including spin degeneracy),
\be
\label{i_lb}
        G \; = \; \frac{2e^2}{h} \mbox{Tr}( t^\dag t) ,
\ee
is no longer convenient because no conventional transverse `lead' states
exist between which to define the transmission matrix $t$.
Rather, a technique based on scattering theory
in the two-dimensional infinite half-plane is appropriate.
We relate conductance to transmission {\em cross section}, defined as an
effective collision size on the reflective boundary of a half-plane
(reservoir) region.
We also introduce a new half-plane radial basis of `lead' states
in which the usual Landauer formula is recovered.

The relation between the Landauer and the half-plane scattering formalism
is expressed by
\be
        \int_{-\pi/2}^{\pi/2} \!\! d\phi \, \sigma_{\rm T}(k,\phi) \; = \; 
        \lambda \mbox{Tr} (t^\dag t),
\ee
where $\sigma_{\rm T}(k,\phi)$ is the angle-dependent transmission cross section
and $\lambda$ the Fermi wavelength,
which I derive for both hard-walled and soft-walled reflective
potential barriers.

We analyse an idealized, highly non-adiabatic slit QPC system
in the extreme quantum, intermediate, and semiclassical regimes.
We derive the
counterintuitive result (first due to Heller) that an 
arbitrarily small (tunneling) QPC can reach a p-wave channel conductance
of $2e^2/h$
when coupled to a suitable resonant cavity.
We also find that if two or more resonances coincide, the
total conductance can in theory reach multiples of this value.

This leads to some thought-experiments on attempting to overcome the maximum
conductance $2e^2/h$ per quantum channel.
We also discuss reciprocity (left-to-right symmetry)
of conductance, and the possibility
of its breakdown in a proposed QPC
(which could exhibit `conductance' quantization) for atom waves\cite{josephqpc}.
We emphasizes the importance of the {\em thermal occupation} of states in
phase space (as is usual in 2D electron systems), for reciprocity to exist.
An analogous atomic QPC in 3D need not have this thermal occupation,
thus in this system reciprocity can be broken.

This work has been in collaboration with Areez Mody and Miriam Blaauboer,
and at the earlier stages many contributions were made by Adam Lupu-Sax.
Joseph Thywissen, and professors Charlie Marcus and Daniel Fisher
also contributed via stimulating discussions.







% 44444444444444444444444444444444444444444444444444444444444444444444444444
\subsubsection{Chapter~\ref{ch:atom}: Waveguides for neutral atoms using
evanescent light fields}

The fourth and final part is a self-contained proposal for a new
design of coherent
atom waveguide, using the forces exerted on atoms by near-resonant
laser fields.
As an independent project, it
does not connect directly with other theoretical work in this thesis.
However it shares many common themes:
the motion of atoms in the trapping potential is a 2D quantum bound mode
problem (in a smooth potential), and
the optical waveguide
bound mode calculation
is also similar to this same problem
(dielectric constant playing the role of a negative potential).
Because the optical `potential' is not hard-walled, and because the fields
are vector rather than scalar,
the efficient methods of Chapter~\ref{ch:ipwdm} or \ref{ch:verg}
do not apply; rather Finite Elements\cite{FEreview} will be used.

There has been much recent
progress
in the
trapping and cooling of neutral atoms, opening up new areas of
ultra-low energy and matter-wave physics \cite{churev,cctrev,billrev}.
Waveguides for such atoms are of great interest for atom optics,
atom interferometery, and atom lithography.
%As with optical guides, the applications of atom waveguides
%fall into the multimode and single-mode regimes.
Multimode atom waveguides act as
incoherent atom pipes
that could
trap atoms, transport them along complicated paths or
between different environments, or deliver
highly localized atom beams to a surface.
% lithography refs?
Single-mode waveguides (or multimode guides populated only by atoms in
the transverse
ground-state) could be used for coherent atom optics and interferometry
\cite{interf,adams}, as well as a tool for one-dimensional physics
such as boson-fermion duality
\cite{joseph,Maxim,1Deffects} and low-dimensional
Bose-Einstein condensation effects \cite{1Dbec}.


In Chapter~\ref{ch:atom} we propose a dipole-force
% line-like, 1D?
linear waveguide which confines neutral
atoms up to $\lambda/2$ above a
microfabricated single-mode dielectric optical guide.
%
The optical guide
%has rectangular section,
carries
far blue-detuned light in the horizontally-polarized TE mode
and
far red-detuned light in the vertically-polarized TM mode,
with both modes close to optical
cut-off.
%
A trapping minimum
in the transverse plane is formed above the optical guide
due to
the differing evanescent
decay lengths of the two modes.
%
This design allows
% multiple
manufacture of
mechanically stable atom-optical elements on a
substrate.


We find that a rectangular optical guide of 0.8\,$\mu$m by 0.2\,$\mu$m
carrying 6\,mW of total laser power
(detuning
$\pm$15\,nm about the D2 line)
gives a trap depth of 
200\,$\mu$K for cesium atoms ($m_F = 0$),
% caesium euro spelling.
transverse oscillation frequencies of $f_x = 40$\,kHz and
$f_y = 160$\,kHz,
collection area $\sim 1\,\mu$m$^{2}$
and coherence time
of 9\,ms.
The laser powers required are orders of magnitude less than those commonly
needed for dipole traps.
The large tranverse frequencies achieved allow the possibility of
atomic single-mode occupation (hence coherent guiding) when
fed from a source at cesium MOT temperature ($\approx 3\mu K$).
We present design equations allowing optimal parameter choices to be made.
We also  discuss the effects of
non-zero $m_F$, the D1 line, surface interactions, heating rate,
the substrate refractive index $n_s$,
and the limits on waveguide bending radius.
It emerges that lowering $n_s$ is the main goal if large trap depths
are desired of order an optical wavelength from the guide surface.

As known in the engineering community, the
optical bound mode problem is notoriously hard \cite{FEreview}.
We calculate the full vector bound modes for an arbitrary guide shape
using two-dimensional
non-uniform finite elements in the frequency-domain,
allowing us to
optimize atom waveguide properties.
We chose rectangular guide cross-sections for this optimization, for simplicity.
There are many other shapes possible; the fabrication technique will
be the determining factor on what is practical.


This work on atom waveguides, an admittedly far-fetched topic
for a student of Rick Heller,
was in fact a collaboration with the following members of the Prentiss Group:
Steve Smith (my principal collaborator),
Maxim Olshanii,
Kent Johnson,
Allan Adams (who introduced me to the problem), and Mara Prentiss.
I also benefitted from discussions with Joseph Thywissen and Yilong Lu.

%%%