\chapter{Conclusions \& Future Work}

As mentioned in the Introduction to this report, the main goals of this report were
to provide a decentralised web-serving network providing the following guarantees:

\begin{enumerate}
    \item{Decentralisation}
    \item{Author Anonymity}
    \item{Reader Privacy}
    \item{Community-Driven Censorship}
    \item{Increased Reliability and Redundancy}
    \item{Secure Data Storage}
\end{enumerate}

These goals were met, and a successful and functional prototype of a peer-to-peer
network was developed, as well as a minimal user-facing front-end which would allow
users to easily interface with the network.

We will now discuss possible future developments for this project, mentioning the most
interesting areas for improvement.

\section{Censorship}

It was determined, however, that there was a trade-off to be made between complete
reader anonymity and accurate censorship. Indeed, if all requests made in the network
are entirely anonymous, then it becomes difficult to prevent abuse by users that
aim to force censorship on certain content. The design of the content-removal algorithm
proposed in Chapter 3 of this report aims to mitigate this possibility for abuse
by using a logarithmic algorithm for content removal. For future work in this
topic, it would be useful to experiment with this algorithm and the values used by
it on a large-scale, which was unfortunately not possible in the short period in which this
project was done.

\section{Cryptography}

Further research into the cryptography used by the system could prove useful.
As homomorphic encryption is developed further and becomes a more viable technology to be
used in real-time, it could be used to transform encrypted data in transit between nodes,
meaning that all of the messages between users could remain totally encrypted.

\section{Exploits}

It would be particularly interesting to investigate possible exploits present in the
system. One such exploit is related to the use of a symmetric-key for encryption,
decryption and MAC calculations. Since this symmetric key is given to users that read
some data on the network, it would be possible for arbitrary users to calculate new
MAC values for data that they retrieve. This attack would only be useful if a user
knew the UUID of a web-page that they wished to exploit, which means that it could
not be performed by a user to exploit the data that they hold, simply because they
would need to brute-force reverse a known content location to a UUID, which would
require for SHA-1 an average of $2^{160} / 2$ SHA-1 calculations.
This potential exploit could be remedied by the use of some cryptographic scheme where
readers would decrypt with a public key, and authors would encrypt (and calculate MACs)
with a different private key.

\section{User Interface}

For realistic use of the system, a more full-featured user front-end would be required.
While the current prototype works well for single pages, a full alternative web-browser
or browser plugin would provide a much better user experience and would allow for easy
inter-page linking and the embedding of content within other pages (such as external
CSS or JavaScript files). The current front-end only provides a prototype, and extra
time spent developing a user-facing application would be useful. Still, the prototype
front-end developed was not the main focus of the project, and the prototype works
well.

\section{Partial File Replication}

More research info the possibility for partial file replication could prove to be
very interesting. Some large, distributed data-stores that have been developed in the
``Big Data'' industry, such as Google's GFS use a ``sharding'' technique, where stored data
is split into ``chunks'' of a particular size and these chunks are distributed across storage
nodes in the network ~\cite{gfs}. These chunks could be partially redundant: one could imagine a system with
$N$ chunks, where only $N-2$ (for example) chunks would be necessary to re-assemble
the original piece of data. This would also save on disk space, and make exploits
on the system more difficult as no single user holds an entire web-page.

\section{Closing}

In conclusion, the system designed in this project provides a functional
proof-of-concept prototype of a secure, anonymous web-serving network with community-driven
censorship.
