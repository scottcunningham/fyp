\chapter{Related Work}

\section{Blanchfield}

Blanchfield, in his paper “An Anonymous and Scalable Peer-to-Peer System” ~\cite{blanchfield},
describes a novel anonymising peer-to-peer network.

Blanchfield defines five types of anonymity:
\begin{enumerate}
    \item{Author anonymity: \\
        A user can not be linked to a document which they have created.
    }
    \item{Publisher anonymity: \\
        A user can not be linked to a document which they have added to the network.
    }
    \item{Reader anonymity: \\
        A user can not be linked to a request to read a document in the network.
    }
    \item{Server anonymity: \\
        A user can not be linked to documents that they are storing.
    }
    \item{Document anonymity: \\
        A user can not know what documents that they are storing.
    }
\end{enumerate}

We will use these definitions throughout the paper.

\section{FreeNet}

In his paper “A Distributed Decentralised Information Storage and Retrieval System” ~\cite{freenet},
Ian Clarke describes a system which he calls “FreeNet”. FreeNet is a distributed, peer-to-peer web
serving system providing document anonymity, author anonymity, and reader anonymity.

FreeNet does not provide any means for censorship. Instead, content on the network is given a
“time to live” value. After this time expires, the content is removed from the network.

\section{Tor, The Onion Router}

Tor ~\cite{tor} is a peer-to-peer web serving network which provides user privacy against web traffic analysis.
It aims to protect its users from malicious third parties analysing their web traffic and using it to
profile users or make assumptions about their browsing habits. It does this through multi-hop routing,
which they refer to as “onion routing”.

Tor takes a strong anti-censorship position. As a consequence, Tor is known to be used for hosting various
illegal content.

\section{Kademlia}

Kademlia ~\cite{kademlia} is a decentralised Distributed Hash Table. It provides a peer-to-peer,
distributed key-value storage network. Data (the `value') can be stored at a network index (called the `key')
and a subsequent request for this key will return this value. Users are identified by a numeric user ID, and
these users store data with keys that are numerically close to this user ID.

Kademlia supports automatic data replication, depending on a global network constant value, called the `replication
constant'. As users join and leave the network, content is automatically distributed to the users that have user IDs
closest to the numeric `key' value for data.`

Kademlia sends queries through its peer-to-peer network in parallel, according to a global network constant value
referred to as \(\alpha\) (`alpha'). When a user receives a request for a certain key, K, it forwards the request to the
\(\alpha\) number of closest user IDs that it knows of. In this way, messages intended to be sent to a certain user ID
are flooded through the network until they reach their intended destination.

\section{BitTorrent}

BitTorrent is a distributed, peer-to-peer file-sharing system ~\cite{torrent}. While BitTorrent is not specifically a
web-serving network, it is a useful example.

On a BitTorrent network, users voluntarily choose content to download. Once portions of data have been
downloaded to a user, they then can provide that content to other users of the network.

Information about content locations are found using a Kademlia-based network, called KAD ~\cite{torrentkad}. Though
content locations in a BitTorrent network are found through the decentralised KAD network,
data transfers are done directly from user to user instead of over the KAD network.

There is no redundancy or reliability built into BitTorrent. Content only stays on a BitTorrent network while
users interested in sharing that content remain on the network.

\section{Elite}

“Elite” is an anonymous, decentralised search engine described by Kilian Levacher in his paper
“Elite, An Ethical Peer-to-Peer Search Engine” ~\cite{levacher}. Elite aims to allow users to hold their own data
instead of it being held by third party search engine providers.

Elite has several interesting design ideas that are of interest to this project:
\begin{itemize}
    \item{Query anonymity through random id replacement}
    \item Human browsing habits to evaluate content (similar to our downvote model)
    \item Wants ability for censorship - quote "enabling peers to democratically make decisions about how information is managed on the system"
    \item Ripple effect - which we can adapt to help with our censorship model
\end{itemize}
