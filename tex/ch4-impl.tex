\chapter{Implementation}

Now that we have discussed the design of our system, we will discuss its implementation.
This design will consist of two main parts:

\begin{enumerate}
	\item{Identical Kademlia user nodes which will make up the Kademlia network.
	The code for these nodes will contain implementation for content look-up, publishing and down-voting and will provide 
	a high-level interface for interfacing with the system. }
	\item A user-facing front-end which will have the ability to look-up content, publish content and down-vote content.
\end{enumerate}

We will first discuss general implementation details, then turn our attention to the implementation of these parts.

\section{The Python programming language}

The project was written using the programming language Python. Python is a high-level interpreted programming language with
light syntax and ``duck-typing'', making it ideal for rapid development. Python also has many useful libraries available,
many of which are open-source under permissive licenses. While none of these factors were essential to the development
of the project, the availability of Python libraries helped with the implementation of the system.

\section{Kademlia Implementations}

While there are several available Python implementations of the Kademlia DHT, many are incomplete or badly documented.
We will discuss the available implementations and their relative advantages and disadvantages.

\subsection{Entangled}

Entangled is a working implementation of the Kademlia DHT written in Python. It provides a high-level programming interface and
a graphical interface for interfacing with the Kademlia network with the ability to demonstrate file-sharing graphically.
Entangled also includes an extra ``delete'' RPC which could be easily modified to serve the purpose of the ``downvote'' RPC
that our system requires. However, Entangled lacks comprehensive documentation and has a complicated API.
Entangled's code-base is complicated and difficult to understand in places. For this reason, it was decided that Entangled
was not suitable to be used for this system.

Entangled is released under the GNU General Public License, v3 (GPLv3).

\subsection{``Kademlia'' Library}

``Kademlia'' a Kademlia DHT implementation in Python. ``Kademlia'' seems to be well-documented and written,
but as its first release first published after the beginning of this project, it was decided that it would be too unstable of
a platform to begin working on. However, the library looks promising and it would be possible for it to be used in future projects
once the code-base has become stable.

``Kademlia'' is released under the MIT license.

\subsection{PyDHT}

PyDHT is a pure Kademlia DHT implementation written in Python. PyDHT does not add extra functionality to the DHT, and its code
is simple enough to be understood, despite its lack of documentation. The code-base of PyDHT is small and modular, making
it possible to easily change the implementation in places. PyDHT includes simple code examples demonstrating how to use the
library. As a result, it was decided that PyDHT would be a suitable base to work on for this project.

PyDHT is released under the BSD 2-clause license.

\subsection{Choice of Kademlia implementation}

It was decided that PyDHT would be used as the base of this project due to its stability, simple API and permissive license.
Since PyDHT is licensed under the 2-clause BSD license, we have forked the library and the modified library will be made
available under the same 2-clause BSD license. The modified library will be made available in Appendix 1.

\section{Cryptography}

\subsection{Key Derivation Functions \& MAC}

The implementation of key-derivation functions and MAC used in the project is written using the PyCrypto Python cryptography library ~\cite{pycrypto}.
The code for this implementation is based on PyCrypto code examples released in to the public domain by Brendan Long ~\cite{kdfexample}.
PyCrypto was chosen due to its proven stability and usage as well as it having functional examples of Key-Derivation Functions.

The key-derivation function chosen to be used for this project was PBKDF2. PBKDF2 was written by RSA labs to replace an earlier standard,
PBKDF1, which could only produce keys with length of up to 160 bits ~\cite{kdf}. PBKDF2 is suitable due to it supporting the generation of
larger keys than 160 bits. This ensures that key-sizes can be increased beyond 160 bits in the future if necessary.
PBKDF2 was chosen due to a readily available implementation being included in the PyCrypto library.
Other key-derivation functions that supported large key-sizes would also be suitable for usage in this system.

The MAC algorithm which was chosen to be used for this project was Hash-based Message Authentication Code (HMAC). HMAC was chosen due to a readily
available implementation being provided in the PyCrypto library and due to it being well-known and proven in the field.
Other MAC algorithms would also be suitable in its place.

\subsection{Hashing function}

The hashing function chosen to be used to derive content locations from UUIDs in the system was SHA-1. This could be replaced with any hashing function
that provided a relatively large key-space. SHA-1 was chosen due to an implementation in the Python standard ``hashlib'' library.

\subsection{Cryptographic keys}

Cryptographic keys are necessary to provide encryption and decryption of data stored by nodes on the network. In order to ensure document anonymity,
the encryption scheme used must be strong enough that it would be infeasible for a node on the network to break it in a reasonable amount of time. The user that publishes some data must be able to encrypt that data, and also allow users to decrypt it.

During the design of the system, we considered both symmetric key encryption and asymmetric key (public-private key) encryption.

Asymmetric-key RSA encryption was initially used for encryption and decryption of content. In RSA, a private-key is used for 
decryption and a public-key for encryption ~\cite{rsa}. Since the system requires that the author encrypts a piece of data and readers decrypt it, the private key would have to be given to users to decrypt web-page data. As a result, asymmetric-key
encryption was decided to be unsuitable for the system.

Symmetric-key encryption uses one cryptographic key for both encryption and decryption. This fits the requirements of our design
specification - the content author and content reader can both use the same cryptographic key for encryption and decryption.
This implementation of the system uses Rijndael AES as its implementation of symmetric-key encryption due to its availability
in the PyCrypto library ~\cite{pycrypto}. Rijndael AES was arbitrarily chosen as it has been proven by its industry use, and had no obvious disadvantage compared to the other algorithms.



pydht modifications
	publish fn
	retrieve fn
	downvote fn